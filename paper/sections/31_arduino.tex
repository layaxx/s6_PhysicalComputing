\subsection{Arduino}
\label{subsec:arduino}

As stated previously, the code that controls the arduino is written in C.
Its purpose is to configure the sensors, monitor them for new data and relay that to the Laptop when available.
Communication with the MPU6050 sensor happens via the Inter-Integrated Circuit (\IIC) protocol.
Communication with the laptop uses a serial connection via USB at a BAUD rate of 250000.


The main control logic for the micro controller is quite simple. Upon startup, all ports and sensors are configured.
Then, a infinite loop is started, which relays sensor data to the laptop if a new value is available.
Availability of new sensor data is determined via interrupts.

For the MPU6050, a interrupt signals that new data is available. In the interrupt handling routine, a variable is the set to one.
In the main loop, this variable is checked, and if it is one, the x value of the gyroscope is read from the mPU via \IIC.
After sending the data to the laptop via the serial interface, the variable is then reset to 0.

For the Ultrasonic sensor, interrupts are used to determine logic level changes on the pin that is connected to the echo signal of the sensor.
Upon activating the sensor, the interrupt is first fired when the echo signal goes high.
Once that happens, the current value of the 16 Bit counter, which is enabled and set to the 8 times prescaler, is saved.
The next time the interrupt is triggered will be when the echo signal goes back down.
When this happens, the timer value is saved again and another variable is set to one. Analogous to reading the MPU,
this variable is used inside the main loop to determine the availability of sensor data. If this variable is set,
the difference of times, i.e. the duration the echo pin has been high, can be sent to the laptop.

To simplify evaluation at the laptop side, values are sent with a prefix indicating the type of sensor,
followed by a colon, a space, the value to be transmitted and a line break sequence.
