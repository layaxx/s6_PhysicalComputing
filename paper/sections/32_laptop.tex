\subsection{Laptop}
\label{subsec:laptop}

The reasons I choose Typescript for the Laptop-side software are that this is the language I have the most experience with and that, when used in the context of a webapp, it facilitates easy visualization of data.
This is especially helpful for debugging, but comes at the price of performance, which I assume is lower than what can be achieved with other languages such as C.

\paragraph{Proxy}
Because a webapp cannot directly access the serial port, I wrote a proxy script, which reads from the serial port and provides received data via the WebSocket protocol.

\paragraph{Detecting a Rotation}

\paragraph{Capturing Ultrasound Data}

\paragraph{Normalizing Data}
Due in part to the variable update rate, which is discussed in Section \ref{subsec:benchmark}, the captured readings need to be normalized.
This is done by using the sum of gyroscope readings starting from the beginning of rotation as x-coordinate for the current distance reading.
The total sum of gyroscope readings for a rotation then corresponds to $180\deg$ and other values can be scaled accordingly.

\paragraph{Evaluating Data}