\subsection{Laptop}
\label{subsec:laptop}

The reasons I choose Typescript for the Laptop-side software are that this is the language I have the most experience with and that, when used in the context of a webapp, it facilitates easy visualization of data.
This is especially helpful for debugging, but comes at the price of performance, which I assume is lower than what can be achieved with other languages such as C.

\paragraph{Proxy}
Because a webapp cannot directly access the serial port, I wrote a proxy script, which reads from the serial port and provides received data via the WebSocket protocol.

\paragraph{Detecting a Rotation}
Detection of a $180\deg$ rotation from left to right is implemented via a state machine.
The relevant sensor values for this detection is the X value of the gyroscope on the MPU6050.
Because the values are not 0 even when the board is completely still, I decided to use the first 50 received values for calibration.
This means that the board should be stable for the first about 2 seconds (see chapter \ref{sec:benchmark}) after connection with the laptop is established.
Those values are used to calculate mean and standard deviation.

If a received value is then within $x$ times the standard deviation of the mean, the platform is considered to not be turning.
If the value is above or below the threshold, the state machine goes to the Over/Under states respectively.
Because outliers are still possible, there are further states, namely Over_Steady and Under_Steady. 
While in the Over/Under states, a counter is maintained, which is increased for each measurement outside the threshold and decreased if the platform is stable again.
If the counter reaches 0, the state returns to the standard, i.e. Steady. If the counter reaches a threshold, the state changes to Over_Steady or Under_Steady depending on previous state.
While these additional states are not currently utilized, they allow to further customize the detection behavior.

While in the Over or Over_Steady states, if value significantly under mean is detected, state is instantly reverted back to Steady and the current rotation ends.

Start of rotation is therefore detected when the state changes away from Steady and end of rotation upon returning to Steady.
For detecting the length of rotation, the sensor values are summed up while state is not Steady.
After a rotation ends, this value can be compared against thresholds to determine what direction this turn was (corresponds to the sign of the sum) and how big the turn was (corresponds to the absolute value of the sum).

% TODO: include visualization of state machine 

\paragraph{Capturing Ultrasound Data}

\paragraph{Normalizing Data}
Due in part to the variable update rate, which is discussed in Section \ref{subsec:benchmark}, the captured readings need to be normalized.
This is done by using the sum of gyroscope readings starting from the beginning of rotation as x-coordinate for the current distance reading.
The total sum of gyroscope readings for a rotation then corresponds to $180\deg$ and other values can be scaled accordingly.
While the gyroscope readings give a speed of rotation rather than a distance, a Sum should still work given the very steady frequency of updates from the MPU (see chapter \ref{benchmark}).
For distance of rotation would be speed of rotation multiplied by time, if we have constant time a simple sum should suffice.
Distances are also scaled to a range between 0 and 1 for the shortest and longest measured distance respectively.

% TODO: describe problem with graphing environment

\paragraph{Evaluating Data}