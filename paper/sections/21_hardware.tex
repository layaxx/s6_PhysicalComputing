\subsection{Hardware}
\label{subsec:hardware}

From a hardware perspective, two main platforms are used: an arduino micro controller which captures the output from the sensors and a laptop, which processes the data.

\paragraph{Arduino}
This consists of an arduino Nano v.3, a MPU6050 used for its gyroscope sensor and an ultrasonic sensor (HCSR04). The wiring can be seen in Figure \ref{fig:schematic}.

\begin{figure}
    \centering
    \includegraphics[width=0.5\linewidth]{figures/schematic_bb.pdf}
    \caption{Wiring of the Sensors.
        Both sensors are mounted upright, with the ultrasonic sensor aimed at the top of the image. Note: this may cause the order of pins to be reversed, wiring in the schematic is according to the labels, not the order.}

    \label{fig:schematic}
\end{figure}


\paragraph{Laptop}
The arduino is connected via mini USB cable to a laptop, which provides the micro controller with power and receives and evaluates sensor data from it.
To enable a smooth rotation, a angled USB cable may be used, but this is not necessary.
While the receiver does not need to be a Laptop, for instance a Desktop computer or even a raspberry pi could be used instead, I will refer to this component as a Laptop in the following sections.